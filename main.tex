\documentclass[11.5pt,a4paper,landscape,oneside]{amsart}
\usepackage{amsmath, amsthm, amssymb, amsfonts}
\usepackage[T1]{fontenc}
\usepackage[english]{babel}
\usepackage[utf8]{inputenc}
\usepackage{booktabs}
\usepackage{caption}
\usepackage{color}
\usepackage{fancyhdr}
\usepackage{float}
\usepackage{fullpage}
%\usepackage{geometry}
% \usepackage[top=0pt, bottom=1cm, left=0.3cm, right=0.3cm]{geometry}
\usepackage[top=3pt, bottom=1cm, left=0.3cm, right=0.3cm]{geometry}
\usepackage{graphicx}
% \usepackage{listings}
\usepackage{subcaption}
\usepackage[scaled]{beramono}
\usepackage{titling}
\usepackage{datetime}
\usepackage{enumitem}
\usepackage{multicol}

\newcommand{\subtitle}[1]{%
  \posttitle{%
    \par\end{center}
    \begin{center}\large#1\end{center}
    \vskip0.1em\vspace{-1em}}%
}

% Minted
\usepackage{minted}
\newcommand{\code}[1]{\inputminted[fontsize=\normalsize,baselinestretch=1]{cpp}{_code/#1}}
\newcommand{\bashcode}[1]{\inputminted{bash}{code/#1}}
\newcommand{\regcode}[1]{\inputminted[xleftmargin=2em,linenos]{cpp}{code/#1}}

\newcommand{\Sequence}[1]{\subsection{#1}}
\newcommand{\Term}[1]{\subsection{#1}}

% Header/Footer
% \geometry{includeheadfoot}
%\fancyhf{}
\pagestyle{fancy}
\lhead{Volgograd State Technical University (Bulankin, Nosov, Penskoy)}
\rhead{\thepage}
\cfoot{}
\setlength{\headheight}{15.2pt}
\setlength{\droptitle}{-20pt}
\posttitle{\par\end{center}}
\renewcommand{\headrulewidth}{0.4pt}
\renewcommand{\footrulewidth}{0.4pt}

% Math and bit operators
\DeclareMathOperator{\lcm}{lcm}
\newcommand*\BitAnd{\mathrel{\&}}
\newcommand*\BitOr{\mathrel{|}}
\newcommand*\ShiftLeft{\ll}
\newcommand*\ShiftRight{\gg}
\newcommand*\BitNeg{\ensuremath{\mathord{\sim}}}
\newcommand{\stirlingii}{\genfrac{\{}{\}}{0pt}{}}
\DeclareRobustCommand{\stirling}{\genfrac\{\}{0pt}{}}

\newenvironment{myitemize}
{ \begin{itemize}[leftmargin=.5cm]
    \setlength{\itemsep}{0pt}
    \setlength{\parskip}{0pt}
    \setlength{\parsep}{0pt}     }
{ \end{itemize}                  }

% Title/Author
\title{Volgograd State Technical University}
\subtitle{Team Reference Document}
\date{\ddmmyyyydate{\today{}}}

% Output Verbosity
\newif\ifverbose
\verbosetrue
% \verbosefalse

\begin{document}

\begin{multicols*}{2}
\maketitle
\thispagestyle{fancy}
\vspace{-3em}

\tableofcontents
\clearpage

\section{Code Templates}
	\subsection{Basic Configuration}
		\subsubsection{.vimrc}
			\bashcode{vimrc.sh}
		\subsubsection{stress and template}
			\regcode{main.cpp}
	\subsection{Vector}
		\regcode{vec.cpp}
	\subsection{FFT}
		\regcode{fft.cpp}
	\subsection{Matrix}
		\regcode{matrix.cpp}
	\subsection{SegmTree}
		\regcode{segm_tree.cpp}
	\subsection{Aho}
		\regcode{aho.cpp}
	\subsection{Suffix Automaton}
		\regcode{suf_auto.cpp}
	\subsection{Stoer Wagner}
		\regcode{stoer_wagner_mincut.cpp}
	\subsection{Flow}
		\regcode{flow.cpp}
	\subsection{Prefix function}
		\regcode{prefix_func.cpp}
	\subsection{BPWS}
		\regcode{bpws.cpp}
	\subsection{Bridge search}
		\regcode{br.cpp}
	\subsection{Lca}
		\regcode{lca.cpp}
	\subsection{2-SAT}
		\regcode{2sat.cpp}
	\subsection{Centroid}
		\regcode{centroid.cpp}

\clearpage

\section{Misc}
        \subsection{Debugging Tips}
            \begin{myitemize}
                \item Stack overflow? Recursive DFS on tree that is actually a long path?
                \item Floating-point numbers
                    \begin{itemize}
                        \item Getting \texttt{NaN}? Make sure \texttt{acos} etc.\ are
                            not getting values out of their range (perhaps
                            \texttt{1+eps}).
                        \item Rounding negative numbers?
                        \item Outputting in scientific notation?
                    \end{itemize}
                \item Wrong Answer?
                    \begin{itemize}
                        \item Read the problem statement again!
                        \item Are multiple test cases being handled correctly?
                              Try repeating the same test case many times.
                        \item Integer overflow?
                        \item Think very carefully about boundaries of all input parameters
                        \item Try out possible edge cases:
                            \begin{itemize}
                                \item $n=0, n=-1, n=1, n=2^{31}-1$ or $n=-2^{31}$
                                \item List is empty, or contains a single element
                                \item $n$ is even, $n$ is odd
                                \item Graph is empty, or contains a single vertex
                                \item Graph is a multigraph (loops or multiple edges)
                                \item Polygon is concave or non-simple
                            \end{itemize}
                        \item Is initial condition wrong for small cases?
                        \item Are you sure the algorithm is correct?
                        \item Explain your solution to someone.
                        \item Are you using any functions that you don't completely understand? Maybe STL functions?
                        \item Maybe you (or someone else) should rewrite the solution?
                        \item Can the input line be empty?
                    \end{itemize}
                \item Run-Time Error?
                    \begin{itemize}
                        \item Is it actually Memory Limit Exceeded?
                    \end{itemize}
            \end{myitemize}

        \subsection{Solution Ideas}
            \begin{myitemize}
                \item Dynamic Programming
                    \begin{itemize}
                        \item Parsing CFGs: CYK Algorithm
                        \item Drop a parameter, recover from others
                        \item Swap answer and a parameter
                        \item When grouping: try splitting in two
                        \item $2^k$ trick
                        \item When optimizing
                            \begin{itemize}
                                \item Convex hull optimization
                                    \begin{itemize}
                                        \item $\mathrm{dp}[i] = \min_{j<i}\{\mathrm{dp}[j] + b[j] \times a[i]\}$
                                        \item $b[j] \geq b[j+1]$
                                        \item optionally $a[i] \leq a[i+1]$
                                        \item $O(n^2)$ to $O(n)$
                                    \end{itemize}
                                \item Divide and conquer optimization
                                    \begin{itemize}
                                        \item $\mathrm{dp}[i][j] = \min_{k<j}\{\mathrm{dp}[i-1][k] + C[k][j]\}$
                                        \item $A[i][j] \leq A[i][j+1]$
                                        \item $O(kn^2)$ to $O(kn\log{n})$
                                        \item sufficient: $C[a][c] + C[b][d] \leq C[a][d] + C[b][c]$, $a\leq b\leq c\leq d$ (QI)
                                    \end{itemize}
                                \item Knuth optimization
                                    \begin{itemize}
                                        \item $\mathrm{dp}[i][j] = \min_{i<k<j}\{\mathrm{dp}[i][k] + \mathrm{dp}[k][j] + C[i][j]\}$
                                        \item $A[i][j-1] \leq A[i][j] \leq A[i+1][j]$
                                        \item $O(n^3)$ to $O(n^2)$
                                        \item sufficient: QI and $C[b][c] \leq C[a][d]$, $a\leq b\leq c\leq d$
                                    \end{itemize}
                            \end{itemize}
                    \end{itemize}
                \item Greedy
                \item Randomized
                \item Optimizations
                    \begin{itemize}
                        \item Use bitset (/64)
                        \item Switch order of loops (cache locality)
                    \end{itemize}
                \item Process queries offline
                    \begin{itemize}
                        \item Mo's algorithm
                    \end{itemize}
                \item Square-root decomposition
                \item Precomputation
                \item Efficient simulation
                    \begin{itemize}
                        \item Mo's algorithm
                        \item Sqrt decomposition
                        \item Store $2^k$ jump pointers
                    \end{itemize}
                \item Data structure techniques
                    \begin{itemize}
                        \item Sqrt buckets
                        \item Store $2^k$ jump pointers
                        \item $2^k$ merging trick
                    \end{itemize}
                \item Counting
                    \begin{itemize}
                        \item Inclusion-exclusion principle
                        \item Generating functions
                    \end{itemize}
                \item Graphs
                    \begin{itemize}
                        \item Can we model the problem as a graph?
                        \item Can we use any properties of the graph?
                        \item Strongly connected components
                        \item Cycles (or odd cycles)
                        \item Bipartite (no odd cycles)
                            \begin{itemize}
                                \item Bipartite matching
                                \item Hall's marriage theorem
                                \item Stable Marriage
                            \end{itemize}
                        \item Cut vertex/bridge
                        \item Biconnected components
                        \item Degrees of vertices (odd/even)
                        \item Trees
                            \begin{itemize}
                                \item Heavy-light decomposition
                                \item Centroid decomposition
                                \item Least common ancestor
                                \item Centers of the tree
                            \end{itemize}
                        \item Eulerian path/circuit
                        \item Chinese postman problem
                        \item Topological sort
                        \item (Min-Cost) Max Flow
                        \item Min Cut
                            \begin{itemize}
                                \item Maximum Density Subgraph
                            \end{itemize}
                        \item Huffman Coding
                        \item Min-Cost Arborescence
                        \item Steiner Tree
                        \item Kirchoff's matrix tree theorem
                        \item Pr\"ufer sequences
                        \item Lov\'asz Toggle
                        \item Look at the DFS tree (which has no cross-edges)
                        \item Is the graph a DFA or NFA?
                            \begin{itemize}
                                \item Is it the Synchronizing word problem?
                            \end{itemize}
                    \end{itemize}
                \item Mathematics
                    \begin{itemize}
                        \item Is the function multiplicative?
                        \item Look for a pattern
                        \item Permutations
                            \begin{itemize}
                                \item Consider the cycles of the permutation
                            \end{itemize}
                        \item Functions
                            \begin{itemize}
                                \item Sum of piecewise-linear functions is a piecewise-linear function
                                \item Sum of convex (concave) functions is convex (concave)
                            \end{itemize}
                        \item Modular arithmetic
                            \begin{itemize}
                                \item Chinese Remainder Theorem
                                \item Linear Congruence
                            \end{itemize}
                        \item Sieve
                        \item System of linear equations
                        \item Values too big to represent?
                            \begin{itemize}
                                \item Compute using the logarithm
                                \item Divide everything by some large value
                            \end{itemize}
                        \item Linear programming
                            \begin{itemize}
                                \item Is the dual problem easier to solve?
                            \end{itemize}
                        \item Can the problem be modeled as a different combinatorial problem? Does that simplify calculations?
                    \end{itemize}
                \item Logic
                    \begin{itemize}
                        \item 2-SAT
                        \item XOR-SAT (Gauss elimination or Bipartite matching)
                    \end{itemize}
                \item Meet in the middle
                \item Only work with the smaller half ($\log(n)$)
                \item Strings
                    \begin{itemize}
                        \item Trie (maybe over something weird, like bits)
                        \item Suffix array
                        \item Suffix automaton (+DP?)
                        \item Aho-Corasick
                        \item eerTree
                        \item Work with $S+S$
                    \end{itemize}
                \item Hashing
                \item Euler tour, tree to array
                \item Segment trees
                    \begin{itemize}
                        \item Lazy propagation
                        \item Persistent
                        \item Implicit
                        \item Segment tree of X
                    \end{itemize}
                \item Geometry
                    \begin{itemize}
                        \item Minkowski sum (of convex sets)
                        \item Rotating calipers
                        \item Sweep line (horizontally or vertically?)
                        \item Sweep angle
                        \item Convex hull
                    \end{itemize}
                \item Fix a parameter (possibly the answer).
                \item Are there few distinct values?
                \item Binary search
                \item Sliding Window (+ Monotonic Queue)
                \item Computing a Convolution? Fast Fourier Transform
                \item Computing a 2D Convolution? FFT on each row, and then on each column
                \item Exact Cover (+ Algorithm X)
                \item Cycle-Finding
                \item What is the smallest set of values that identify the solution? The cycle structure of the permutation? The powers of primes in the factorization?
                \item Look at the complement problem
                    \begin{itemize}
                        \item Minimize something instead of maximizing
                    \end{itemize}
                \item Immediately enforce necessary conditions. (All values greater than 0? Initialize them all to 1)
                \item Add large constant to negative numbers to make them positive
                \item Counting/Bucket sort
            \end{myitemize}
	\clearpage
	\section{Formulas}
	\Term{Abel tram}
	\begin{equation}
		\sum_{k=0}^n a_k b_k=a_n B_n - \sum_{k=0}^{n-1} B_k (a_{k+1} - a_k),\quad\text{where}\ B_n = \sum_{k=0}^n b_k
	\end{equation}
	\Term{arithmetic geometric progression}
	\begin{equation}
		u_{n}=qu_{n-1}+d,\quad\text{where}\ q\ne 1, d\ne 0
	\end{equation}
	\begin{equation}
		u_{n+1} = q^{n}\left(u_{1}+\frac{d}{q-1}\right)- \frac{d}{q-1}
	\end{equation}
	\begin{equation}
		\lim\limits_{n\to\infty}u_{n}=\frac{d}{1-q}\quad\text{given}\ \lvert q\rvert<1
	\end{equation}
	\begin{equation}
		\sum_{k=1}^{n}u_{k} = \frac{(u_{1}(q-1)+d)(q^n-1)}{(q-1)^2}-\frac{dn}{q-1}
	\end{equation}
	\Term{Golden section}
	\begin{equation}
		\varphi=\frac{1+\sqrt{5}}{2}
	\end{equation}
	\Sequence{Fibonachi}
	\begin{equation}
		F_0=0,\quad F_1=0,\quad F_{n}=F_{n-1}+F_{n-2}
	\end{equation}
	0, 1, 1, 2, 3, 5, 8, 13, 21, 34, 55, 89, 144, 233, 377, 610, 987, 1597, 2584, 4181, 6765, 10946, 17711, 28657, 46368, 75025, 121393, 196418, 317811, 514229, 832040, 1346269, 2178309, 3524578, 5702887, 9227465, 14930352, 24157817, 39088169, 63245986, 102334155
	\begin{equation}
		F_{n}=\frac{\varphi^n - (-\varphi)^{-n}}{\varphi - (-\varphi)^{-1}}=\frac{\left(\frac{1 + \sqrt{5}}{2}\right)^n - \left(\frac{1-\sqrt{5}}{2}\right)^n}{\sqrt{5}} = \frac{(1+\sqrt{5})^n-(1-\sqrt{5})^n}{2^n\cdot\sqrt{5}}
	\end{equation}
	\begin{equation}
		F_n = \left\lfloor\frac{\varphi^n}{\sqrt{5}}\right\rceil=\left\lfloor\frac{\varphi^n}{\sqrt{5}}+\frac{1}{2}\right\rfloor
	\end{equation}
	\begin{equation}
		n(F) = \left\lfloor \log_\varphi \left(F\cdot\sqrt{5} + \frac{1}{2}\right) \right\rfloor
	\end{equation}
	\begin{equation}
		F_n\sim\frac{\varphi^n}{\sqrt{5}}\quad\text{given}\ n\to\infty
	\end{equation}
	\begin{equation}
		\lim_{n\to\infty}\frac{F_{n+1}}{F_n}=\varphi
	\end{equation}
	\begin{equation}
		F_{n+1}=\sum_{k=0}^{\left\lfloor\frac{n}{2}\right\rfloor} \binom{n-k}{k}
	\end{equation}
	\begin{equation}
		\sum_{i=1}^n F_i = F_{n+2} - 1
	\end{equation}
	\begin{equation}
		\sum_{i=0}^{n-1} F_{2i+1} = F_{2n}
	\end{equation}
	\begin{equation}
		\sum_{i=1}^{n} F_{2i} = F_{2n+1}-1
	\end{equation}
	\begin{equation}
		\sum_{i=1}^n {F_i}^2 = F_{n} F_{n+1}
	\end{equation}
	\begin{equation}
		F_{n-1}F_{n+1} - F_n^2 = (-1)^n
	\end{equation}
	\begin{equation}
		F_n^2 - F_{n-r}F_{n+r} = (-1)^{n-r}F_r^2
	\end{equation}
	\begin{equation}
		F_m F_{n+1} - F_{m+1} F_n = (-1)^n F_{m-n}
	\end{equation}
	\begin{equation}
		F_{2n} = F_{n+1}^2 - F_{n-1}^2 = F_n \left(F_{n+1}+F_{n-1}\right) = F_n \left(F_{n}+2F_{n-1}\right)
	\end{equation}
	\begin{equation}
		F_{2n+1} = F_{n}^2 + F_{n+1}^2
	\end{equation}
	\Sequence{Tribonachi}
	\begin{equation}
		t_0=0,\quad t_1=0,\quad t_2=1,\quad t_{n}=t_{n-1}+t_{n-2}+t_{n-3}
	\end{equation}
	0, 0, 1, 1, 2, 4, 7, 13, 24, 44, 81, 149, 274, 504, 927, 1705, 3136, 5768, 10609, 19513, 35890, 66012, 121415, 223317, 410744, 755476, 1389537, 2555757, 4700770, 8646064, 15902591, 29249425, 53798080, 98950096, 181997601, 334745777, 615693474, 1132436852
	\Sequence{Luk numbers}
	\begin{equation}
		L_0=2,\quad L_1=1,\quad L_{n}=L_{n-1}+L_{n-2}
	\end{equation}
	2, 1, 3, 4, 7, 11, 18, 29, 47, 76, 123, 199, 322, 521, 843, 1364, 2207, 3571, 5778, 9349, 15127, 24476, 39603, 64079, 103682, 167761, 271443, 439204, 710647, 1149851, 1860498, 3010349, 4870847, 7881196, 12752043, 20633239, 33385282, 54018521, 87403803
	\begin{equation}
		L_n = F_{n-1}+F_{n+1}=F_n+2F_{n-1} = F_{n+2}-F_{n-2}
	\end{equation}
	\begin{equation}
		L_n = \varphi^n + (1-\varphi)^{n} = \varphi^n + (- \varphi)^{-n}=\left({ 1+ \sqrt{5} \over 2}\right)^n + \left({ 1- \sqrt{5} \over 2}\right)^n
	\end{equation}
	\Sequence{Binomial coefficients}
	\begin{equation}
		(x+y)^n=\sum_{k=0}^n\binom{n}{k} x^{n-k}y^k
	\end{equation}
	\begin{equation}
		\binom{n}{k} = \frac{n!}{k!\cdot(n-k)!}
	\end{equation}
	\begin{equation}
		\binom{n}{k} = \binom{n-1}{k-1} + \binom{n-1}{k}
	\end{equation}
	\begin{equation}
		\binom{n}{k} = \binom{n}{n-k}
	\end{equation}
	\begin{equation}
		\binom{n}{k} = \frac{n}{k}\cdot\binom{n-1}{k-1}
	\end{equation}
	\begin{equation}
		\binom{n}{m}\binom{n-m}{k}=\binom{n}{k}\binom{n-k}{m}
	\end{equation}
	\begin{equation}
		\sum_{j=0}^k\binom{m}{j}\binom{n-m}{k-j}=\binom{n}{k}
	\end{equation}
	\begin{equation}
		\sum_{j=k}^n\binom{j}{k}=\binom{n+1}{k+1}
	\end{equation}
	\begin{equation}
		\sum_{k=0}^n \binom{n}{k} = 2^n
	\end{equation}
	\begin{equation}
		\sum_{k=0}^n \binom{n}{k}^2 = \binom{2n}{n}
	\end{equation}
	\begin{equation}
		\sum_{k=0}^n\binom{n}{k}^3=\sum_{k=0}^n\binom{n}{k}^2\binom{2k}n
	\end{equation}
	\begin{equation}
		\sum_{k=0}^n (-1)^k\binom{n}{k} = 0
	\end{equation}
	\begin{equation}
		\sum_{k=0}^{2n}(-1)^k\binom{2n}{k}^3=(-1)^n\cdot\frac{(3n)!}{(n!)^3}
	\end{equation}
	\begin{equation}
		\sum_{k=0}^m (-1)^k\binom{n}{k} = (-1)^m\binom{n-1}{m}
	\end{equation}
	\begin{equation}
		\sum_{k=m}^n \binom{n}{k}\binom{k}{m} = 2^{n-m}\binom{n}{m}
	\end{equation}
	\begin{equation}
		\sum_{k=0}^n k \binom{n}{k} = n 2^{n-1}
	\end{equation}
	\begin{equation}
		\sum_{k=0}^n k^2 \binom{n}{k} = n(n + 1)2^{n-2}
	\end{equation}
	\Sequence{Catalan numbers}
	\begin{equation}
		C_0 = 1,\quad C_{n+1}=\sum_{i=0}^{n}C_i\cdot C_{n-i}\
	\end{equation}
	1, 1, 2, 5, 14, 42, 132, 429, 1430, 4862, 16796, 58786, 208012, 742900, 2674440, 9694845, 35357670, 129644790, 477638700, 1767263190, 6564120420, 24466267020, 91482563640, 343059613650, 1289904147324, 4861946401452, 18367353072152, 69533550916004, 263747951750360, 1002242216651368, 3814986502092304
	\begin{equation}
		C_n = \frac{1}{n+1}\binom{2n}{n} = \frac{(2n)!}{(n+1)!\cdot n!} = \prod_{k=2}^{n}\frac{n+k}{k}
	\end{equation}
	\begin{equation}
		C_{n+1}=\frac{2(2n+1)}{n+2}\cdot C_n
	\end{equation}
	\Sequence{Stirling numbers 1}
	\begin{equation}
		\stirlingii{0}{0} = 1,\quad\stirlingii{n}{0}=\stirlingii{0}{n}=0,\quad\stirlingii{n+1}{k} = n \stirlingii{n}{k} + \stirlingii{n}{k-1}
	\end{equation}
	\begin{equation}
		\stirlingii{n}{1} = (n-1)!,\quad\stirlingii{n}{n-1}=\binom{n}{2},\quad\stirlingii{n}{n}=1
	\end{equation}
	\begin{equation}
		\sum_{k=0}^n\stirlingii{n}{k} = n!
	\end{equation}
	\begin{equation}
		\prod_{j=0}^{n-1}(x+j)=\sum_{k=0}^n \stirlingii{n}{k}x^k
	\end{equation}
	\begin{equation}
		\prod_{j=0}^{n-1}(x-j)=\sum_{k=0}^n (-1)^{n-k}\stirlingii{n}{k}x^k
	\end{equation}
	\Sequence{Stirling numbers 2}
	\begin{equation}
		\stirlingii{0}{0} = 1,\quad\stirlingii{n}{0}=\stirlingii{0}{n}=0,\quad\stirlingii{n+1}{k} = k \stirlingii{n}{k} + \stirlingii{n}{k-1}
	\end{equation}
	\begin{equation}
		\stirlingii{n}{1} = 1,\quad\stirlingii{n}{2} = 2^{n-1}-1,\quad\stirlingii{n}{n-1}=\binom{n}{2},\quad\stirlingii{n}{n}=1
	\end{equation}
	\begin{equation}
		\sum_{k=0}^n \stirlingii{n}{k}\prod_{j=0}^{n-1}(x-j)=x^n
	\end{equation}
	\begin{equation}
		\stirlingii{n}{k}=\frac{1}{k!}\sum_{j=0}^k (-1)^{k-j}\binom{k}{j}j^n
	\end{equation}
	\begin{equation}
		\stirlingii{n+1}{k+1} = \sum_{j=k}^n \binom{n}{j} \stirlingii{j}{k}
	\end{equation}
	\begin{equation}
		\stirlingii{n+1}{k+1} = \sum_{j=k}^n (k+1)^{n-j} \stirlingii{j}{k}
	\end{equation}
	\begin{equation}
		\stirlingii{n+k+1}{k} = \sum_{j=0}^k j \stirlingii{n+j}{j}
	\end{equation}
	\Sequence{Bell numbers}
	\begin{equation}
		B_0=1,\quad B_{n+1}=\sum_{k=0}^{n} \binom{n}{k} B_k
	\end{equation}
	1, 1, 2, 5, 15, 52, 203, 877, 4140, 21147, 115975, 678570, 4213597, 27644437, 190899322, 1382958545, 10480142147, 82864869804, 682076806159,  5832742205057, 51724158235372, 474869816156751, 4506715738447323, 44152005855084346, 445958869294805289, 4638590332229999353, 49631246523618756274
	\begin{equation}
		B_n=\sum_{k=0}^n \stirlingii{n}{k}
	\end{equation}
		\clearpage
		\begin{tabular}{@{}l|l|l@{}}
    \toprule
    Catalan	&	$C_0=1$, $C_n=\frac{1}{n+1}\binom{2n}{n} = \sum_{i=0}^{n-1}C_iC_{n-i-1} = \frac{4n-2}{n+1}C_{n-1}$  & \\
    Stirling 1st kind & $\left[{0\atop 0}\right]=1$, $\left[{n\atop 0}\right]=\left[{0\atop n}\right]=0$, $\left[{n\atop k}\right]=(n-1)\left[{n-1\atop k}\right]+\left[{n-1\atop k-1}\right]$ & \#perms of $n$ objs with exactly $k$ cycles\\
    Stirling 2nd kind & $\left\{{n\atop 1}\right\}=\left\{{n\atop n}\right\}=1$, $\left\{{n\atop k}\right\} = k \left\{{ n-1 \atop k }\right\} + \left\{{n-1\atop k-1}\right\}$ & \#ways to partition $n$ objs into $k$ nonempty sets\\
    Euler	& $\left \langle {n\atop 0} \right \rangle = \left \langle {n\atop n-1} \right \rangle = 1 $, $\left \langle {n\atop k} \right \rangle = (k+1) \left \langle {n-1\atop {k}} \right \rangle + (n-k)\left \langle {{n-1}\atop {k-1}} \right \rangle$ & \#perms of $n$ objs with exactly $k$ ascents \\
    Euler 2nd Order &  $\left \langle \!\!\left \langle {n\atop k} \right \rangle \!\! \right \rangle = (k+1) \left \langle \!\! \left \langle {{n-1}\atop {k}} \right \rangle \!\! \right \rangle +(2n-k-1)\left \langle \!\! \left \langle {{n-1}\atop {k-1}} \right \rangle  \!\! \right \rangle$ & \#perms of ${1,1,2,2,...,n,n}$ with exactly $k$ ascents \\
    Bell & $B_1 = 1$, $B_n = \sum_{k=0}^{n-1} B_k \binom{n-1}{k} = \sum_{k=0}^n\left\{{n\atop k}\right\}$ & \#partitions of $1..n$ (Stirling 2nd, no limit on k)\\
    \bottomrule
    \end{tabular}

    \vspace{10pt}
    \begin{tabular}{ll}
        \#labeled rooted trees & $n^{n-1}$ \\
        \#labeled unrooted trees & $n^{n-2}$ \\
        \#forests of $k$ rooted trees & $\frac{k}{n}\binom{n}{k}n^{n-k}$ \\
        % Kirchoff's theorem
        $\sum_{i=1}^n i^2 = n(n+1)(2n+1)/6$ & $\sum_{i=1}^n i^3 = n^2(n+1)^2/4$ \\
        $!n = n\times!(n-1)+(-1)^n$ & $!n = (n-1)(!(n-1)+!(n-2))$ \\
        $\sum_{i=1}^n \binom{n}{i} F_i = F_{2n}$ & $\sum_{i} \binom{n-i}{i} = F_{n+1}$ \\
        $\sum_{k=0}^n \binom{k}{m} = \binom{n+1}{m+1}$ & $x^k = \sum_{i=0}^k i!\stirling{k}{i}\binom{x}{i} = \sum_{i=0}^k \left\langle {k \atop i} \right\rangle\binom{x+i}{k}$ \\

        $a\equiv b\pmod{x,y} \Rightarrow a\equiv b\pmod{\lcm(x,y)}$ & $\sum_{d|n} \phi(d) = n$ \\
        $ac\equiv bc\pmod{m} \Rightarrow a\equiv b\pmod{\frac{m}{\gcd(c,m)}}$ & $(\sum_{d|n} \sigma_0(d))^2 = \sum_{d|n} \sigma_0(d)^3$ \\
        $p$ prime $\Leftrightarrow (p-1)!\equiv -1\pmod{p}$ & $\gcd(n^a-1,n^b-1) = n^{\gcd(a,b)}-1$ \\
        $\sigma_x(n) = \prod_{i=0}^{r} \frac{p_i^{(a_i + 1)x} - 1}{p_i^x - 1}$ & $\sigma_0(n) = \prod_{i=0}^r (a_i + 1)$ \\
        $\sum_{k=0}^m (-1)^k \binom{n}{k} = (-1)^m \binom{n-1}{m}$ & \\
        $2^{\omega(n)} = O(\sqrt{n})$ & $\sum_{i=1}^n 2^{\omega(i)} = O(n \log n)$ \\
        % Kinematic equations
        $d = v_i t + \frac{1}{2}at^2$ & $v_f^2 = v_i^2 + 2ad$ \\
        $v_f = v_i + at$ & $d = \frac{v_i + v_f}{2}t$ \\
    \end{tabular}
    \subsection{The Twelvefold Way}
        Putting $n$ balls into $k$ boxes.\\
    \begin{tabular}{@{}c|c|c|c|c|l@{}}
    Balls & same & distinct & same & distinct & \\
    Boxes & same & same & distinct & distinct & Remarks\\
    \hline
        - & $\mathrm{p}_k(n)$ & $\sum_{i=0}^k \stirling{n}{i}$ & $\binom{n+k-1}{k-1}$ & $k^n$ & $\mathrm{p}_k(n)$: \#partitions of $n$ into $\le k$ positive parts \\
        $\mathrm{size}\ge 1$ & $\mathrm{p}(n,k)$ & $\stirling{n}{k}$ & $\binom{n-1}{k-1}$ & $k!\stirling{n}{k}$ & $\mathrm{p}(n,k)$: \#partitions of $n$ into $k$ positive parts \\
        $\mathrm{size}\le 1$ & $[n \le k]$ & $[n \le k]$ & $\binom{k}{n}$ & $n!\binom{k}{n}$ & $[cond]$: $1$ if $cond=true$, else $0$\\
    \bottomrule
    \end{tabular}
\clearpage
    \section*{Practice Contest Checklist}
        \begin{itemize}
            \item How many operations per second? Compare to local machine.
            \item What is the stack size?
            \item How to use printf/scanf with long long/long double?
            \item Are \texttt{\_{}\_{}int128} and \texttt{\_{}\_{}float128} available?
            \item Does MLE give RTE or MLE as a verdict? What about stack overflow?
            \item What is \texttt{RAND\_{}MAX}?
            \item How does the judge handle extra spaces (or missing newlines) in the output?
            \item Look at documentation for programming languages.
            \item Try different programming languages: C++, Java and Python.
            \item Try the submit script.
            \item Try local programs: i?python[23], factor.
            \item Try submitting with \texttt{assert(false)} and \texttt{assert(true)}.
            \item Return-value from \texttt{main}.
            \item Look for directory with sample test cases.
            \item Make sure printing works.

            \item Remove this page from the notebook.
        \end{itemize}

\end{multicols*}
\end{document}
